\section*{Technical Dimensions of Programming Systems: major revisions}

We thank the reviewers for their continued attention and feedback. The
major changes we have made are as follows.

\paragraph{Definition of ``system''.}

We worked out a sensible explicit definition of what a programming
system is and included it as early as possible in the paper.

\paragraph{L/O/A typology.}

We intended the three-way classification in section~3 merely to provide
some rhyme and reason as to where our example systems were coming from.
We did not intend it as a rigid scheme where membership in a category
means we are making some profound or bold claim. The L/O/A-type naming
seems to have misled readers into anticipating the latter, so we have
relaxed the naming to ``language-based'', ``OS-like'' and
``application-focused''. We did refer back to this scheme in one
specific context, in section~5.2, noting that most OS-likes (Unix, Lisp
and Smalltalk) cluster in a region to the right.

\paragraph{Dark \& evaluation.}

Section~5 has been reframed as an evaluation of the framework by means
of the two example usages. We've included a summary table for the
dimensional analysis of Dark in section~5.1 and annotated the screenshot
for convenience. We appreciate the reviewer's desire for examples of the
specific interactions. We feel these are best demonstrated through one
of Dark's tutorial videos, so we added a reference to one that addresses
this concern.

\paragraph{References to strengthen claims.}

Our statement about ``vaguely interesting'' was entirely from
experience, so we relaxed it to something we could more easily support
with references. These can be seen in a footnote with our remarks
linking them to the point we're trying to make. Elsewhere we have
addressed this by deleting the claim (e.g.~for ``holistic view'') or
adding references, for example to the various example systems like Java
and Jupyter. As for explicit examples of venues for publishing
programming systems research: we already listed several examples in
section~2 (UIST, VL/HCC, LIVE, PX) and while further venues could be
listed, we believe these illustrate the range well enough. If this was
about specific publications instead, we've now referred to some in the
footnote on page 3.

\paragraph{Socio-Technical dimensions?}

Almost all of our 22 dimensions are straightforwardly technical, with
the possible exception of Learnability and Sociability being
socio-technical, and some remarks in Integrity vs.~Openness. Even though
there are some social aspects, we hope to emphasise the technical side.
On this basis, we would rather not call the contribution
``socio-technical'' dimensions or similar, as this would give a false
expectation to readers. Our preferred naming remains ``technical''---we
put a lot of effort into thinking of names early on, but ``technical
dimensions'' was the best out of perhaps only two viable choices. If
this is unacceptable, we could fall back on the other name, ``design
dimensions'', but if that doesn't work we would need some help
determining an alternative.

\paragraph{On the new length of the paper.}

Following approval from the editors, this revision includes the full
catalogue of dimensions in the main body of the paper. While we
acknowledge this makes it very long, we do not intend reviewers or
readers to read this section from start to finish. Instead, it is meant
as a reference to be used as needed, for example to get further detail
on the dimensions we reference later for the Dark programming system.
This lets us present the material as we originally intended and
addresses a reviewer concern that it was awkward to refer to appendix
material in the section on Dark. In summary, we always saw the
dimensions catalogue as belonging in the main body, but for review
purposes it may still be treated like an appendix.
