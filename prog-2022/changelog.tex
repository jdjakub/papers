\section*{Technical Dimensions of Programming Systems: major revisions}

We graciously thank the reviewers for their detailed feedback. We've
made many minor changes to the paper and several major changes to both
structure and content.

Two issues particularly stood out:

\paragraph{Definition of ``system''.}

All three reviewers requested clarification on what constitutes a
``system'', mentioning IDEs amongst other things as having unclear
membership. Unfortunately, we truly don't have a definition behind the
scenes which we're using to include or exclude examples. It's more of a
``family resemblance'' between things that generalize ``programming
languages'' to ``software which lets one program, yet doesn't have to be
a language''. Nevertheless, we thought it might be worth adding a new
classification scheme for the examples we mention, which should get
across the key ``reference classes'' from which we're drawing. In sum,
these are: things that are like operating systems (O-type), things that
are like domain-specific applications (A-type) and the systems that are
based around a programming language (L-type). We also went into detail
about how a programming language (Java) can be interpreted as a system
in several different ways. These were all additions to Section 3.

\paragraph{But how do you use it?}

Two reviewers were concerned that it's hard to see how to use the
framework. We've rectified this by including a new, concrete
``Discussion'' section with two halves: one using the framework for
analysis, the other for synthesis. Specifically, we analyze Dark, a
recent programming system, along many of the dimensions. Then, we plot
two dimensions (self-sustainability vs.~notational diversity) against
each other to reveal an unexplored gap. The methodology we used to turn
qualitative dimensions into numbers is given in a new Appendix. Various
points earlier in the paper (e.g.~the Abstract and Figure 1) have been
updated to set up for this new section.

Other changes:

\begin{itemize}
\tightlist
\item
  We've significantly reworked Section 3 to be more to-the-point about
  the example systems and less of a historical narrative.
\item
  We've argued for why we think ``sociability'' belongs as a technical
  dimension in the relevant section.
\item
  We've reorganized and re-worded parts of Section 2. We've also made
  the part about how the framework relates to Cognitive Dimensions in
  Section 2, under ``Already-known characteristics'', more explicit.
\item
  We've clarified some of the points about Direct Manipulation
  (primitive actions) and Immediate Feedback (pure code previews).
\item
  Under ``Addressing and Externalizability'', we've made it explicit
  that CSS provides Additive Authoring for the Web programming system.
\item
  To make room for the ``Discussion'' section, we've had to move more
  material to the existing Appendix with the bulk of the dimensions.
  Now, the main body only contains the Customizability dimensions.
\item
  We've elevated the singleton Automation cluster to a doublet by adding
  a Factoring of Complexity dimension.
\end{itemize}
