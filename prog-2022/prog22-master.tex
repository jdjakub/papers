\documentclass[english,submission]{programming}

%\citestyle{acmauthoryear}
%\markboth{}{}

% Class scrartcl Warning: seems someone has broken package `auxhook'.
% Usually this happens, if `auxhook' is loaded or used implicitly or explicitly
% by patching \document or via etoolbox command \AtEndPreamble. Trying an
% emergency workaround. You can avoid this warning adding:
\usepackage{auxhook}
% before \begin{document} on input line 6.
\usepackage{changepage} % For the single-page summary table PDF insert.
\usepackage{pdfpages}
\usepackage{pifont} % For checkmark / cross symbols for Appendix table :)
\usepackage{amsthm} % For numberless Definition
\usepackage{epigraph} % For the opening quotes

% Thanks https://tex.stackexchange.com/a/32687
\NewDocumentCommand{\rot}{O{45} O{1em} m}{\makebox[#2][l]{\rotatebox{#1}{#3}}}%

\begin{document}

\paperdetails{perspective=art, area={Programming systems}}

\title{Technical Dimensions of Programming Systems}

\author{Joel Jakubovic}
\affiliation{
University of Kent, Canterbury, UK
\email{jdj9@kent.ac.uk}
}
\author{Jonathan Edwards}
\affiliation{\email{jonathanmedwards@gmail.com}}
\author{Tomas Petricek}
\affiliation{
University of Kent, Canterbury, UK
\email{T.Petricek@kent.ac.uk}
}

\keywords{Programming Systems, Dimensions, Design, Framework, Analysis}

% Please go to https://dl.acm.org/ccs/ccs.cfm and generate your Classification
% System [view CCS TeX Code] stanz and copy _all of it_ to this place.
%% From HERE
\begin{CCSXML}
<ccs2012>
   <concept>
       <concept_id>10011007.10011006.10011066.10011069</concept_id>
       <concept_desc>Software and its engineering~Integrated and visual development environments</concept_desc>
       <concept_significance>500</concept_significance>
       </concept>
   <concept>
       <concept_id>10003120.10003121.10003129</concept_id>
       <concept_desc>Human-centered computing~Interactive systems and tools</concept_desc>
       <concept_significance>300</concept_significance>
       </concept>
   <concept>
       <concept_id>10003120.10003121.10003122</concept_id>
       <concept_desc>Human-centered computing~HCI design and evaluation methods</concept_desc>
       <concept_significance>300</concept_significance>
       </concept>
 </ccs2012>
\end{CCSXML}
\ccsdesc[500]{Software and its engineering~Integrated and visual development environments}
\ccsdesc[300]{Human-centered computing~Interactive systems and tools}
\ccsdesc[300]{Human-centered computing~HCI design and evaluation methods}

\maketitle

\begin{abstract}
  \emph{Context.} Programming requires much more than just writing code in a programming language. It is usually done in the context of a stateful environment, by interacting with a system through a graphical user interface. Yet, this wide space of possibilities lacks a common structure for navigation. Work on programming systems fails to form a coherent body of research, making it hard to improve on past work and advance the state of the art.
  
  \emph{Inquiry.} In computer science, much has been said and done to allow comparison of \emph{programming languages}, yet no similar theory exists for \emph{programming systems;} we believe that programming systems deserve a theory too. 
  
  \emph{Approach.} We present a framework of \emph{technical dimensions} which capture the underlying characteristics of programming systems and provide a means for conceptualizing and comparing them. 
  
  \emph{Knowledge.} We identify technical dimensions by examining past influential programming systems and reviewing their design principles, technical capabilities, and styles of user interaction. Technical dimensions capture characteristics that may be studied, compared and advanced independently. This makes it possible to talk about programming systems in a way that can be shared and constructively debated rather than relying solely on personal impressions.
   
  \emph{Grounding.} Our framework is derived using a qualitative analysis of past programming systems. We outline two concrete ways of using our framework. First, we show how it can analyze a recently developed novel programming system. Then, we use it to identify an interesting unexplored point in the design space of programming systems. 
  
  \emph{Importance.} Much research effort focuses on building programming systems that are easier to use, accessible to non-experts, moldable and/or powerful, but such efforts are disconnected. They are informal, guided by the personal vision of the authors and thus are only evaluable and comparable on the basis of individual experience using them. By providing foundations for more systematic research, we can help programming systems researchers to stand, at last, on the shoulders of giants.

NOTE TO REVIEWERS: The main contribution of the paper is a comprehensive survey of 22 design dimensions of programming systems. Each one comes with detailed discussion, including known uses, to properly motivate it. Regrettably, we only have space for a small number of these dimensions in the main paper body. To give a broader picture of what we are contributing, we invite the reader to explore Appendix\ \ref{dimensions-catalogue} for dimensions that suit their interest.
\end{abstract}

%\thispagestyle{empty}

\newcommand{\joel}[1]{}
\newcommand{\note}[1]{}
\newcommand{\tp}[1]{}
\newcommand{\mybox}[1]{\noindent\fbox{\parbox{\textwidth}{#1}}}
\providecommand{\tightlist}{}% Don't want Pandoc's tight lists
\newtheorem*{defn}{Definition}
%\newcommand{\hypertarget}[1]{}

\input{prog22-manuscript.tex}

\appendix
\input{prog22-appendix.tex}

\bibliography{prog22}
\end{document}
