\begin{table}
%\centering
\newcommand{\summ}[1]{\multicolumn{2}{p{0.7\linewidth}|}{#1} \\ \hline}

\setlength\extrarowheight{2pt}
\begin{adjustwidth}{-2cm}{-2cm}
\begin{tabular}{| >{\raggedleft\arraybackslash}p{0.25\linewidth} | p {0.35\linewidth} | >{\raggedright\arraybackslash}p{0.35\linewidth} |}
\hline
Dimension (CLUSTER) & Summary & Range of examples \\ \hline
\hline INTERACTION &
\summ{How do users execute their ideas, evaluate the result, and generate new ideas in response?}
\def\arraystretch{1}
Feedback Loops &
How wide are the various gulfs of \emph{execution} and \emph{evaluation} and how do they relate to each other? &
Immediate Feedback (short) vs batch mode (long) evaluation \\
Modes of Interaction &
TODO &
TODO \\
Abstraction Construction &
How do we go from abstractions to concrete examples and vice versa? &
TODO \\
\hline NOTATION &
\summ{How are the different textual / visual programming notations related?}
Notational Structure &
What notations are used to program the system and how are they related? &
Notations overlap and need sync vs. complement each other \\
Surface / Internal Notations &
What is the connection between what a user sees and what a computer program sees? &
Sequence Editing vs. Rendering, Structure Editing vs. Recovery \\
Primary / Secondary Notations &
Is one notation more important than others? &
Build scripts are secondary vs. equal importance of visual editor and code in Flash \\
Expression Geography &
Do similar expressions encode similar programs? &
Concise yet error-prone vs. explicit yet verbose \\
Uniformity of Notations &
Does the notation use a small or a large number of basic concepts? &
Lisp S-expressions vs. English-like textual notations \\
\hline ARCHITECTURE &
\summ{What is the system's attitude to its own organization and that of the outside world?}
Conceptual Outlook &
Does the system present as \emph{designed} or \emph{improvised}? &
Integrity (Everything is a X) vs. openness (compatible mixtures) \\
Composability &
What are the primitives? How can they be combined to achieve novel behaviors? &
TODO \\
Convenience &
Which wheels do users not need to reinvent? &
TODO \\
Commonality &
How much is \emph{common structure} explicitly marked as such? &
Common structure is redundantly flattened vs. factored out\\
\hline CUSTOMIZABILITY &
\summ{Once a program exists in the system, how can it be extended and modified?}
Staging of Customization &
Must we customize \emph{running} programs differently to \emph{inert} ones? Do these changes last beyond termination? &
TODO \\
Addressability &
How many actual and potential \emph{system states} can customizations refer to? &
TODO \\
Additive Authorship &
How far can the system's behavior be changed by \emph{adding} expressions? &
TODO \\
Self-Sustainability &
How far can the system’s behavior be changed from within? &
TODO \\
\hline AUTOMATION &
\summ{To what extent, and in what ways, does the system remove the need to spell out implementation in minute detail?}
Degrees of Automation &
What part of program logic does not need to be explicitly specified? &
Automatic memory management is a low-end example; Prolog a higher-end one. \\
\hline ERRORS &
\summ{What does the system consider to be an error, and how does it approach their prevention and handling?}
Error Detection &
What errors can be detected in which feedback loops and what is the detection mechanism? &
In live coding, errors are detected by humans; in static typing, this is partly automated. \\
Error Response &
How does the system respond when an error is detected? &
Does it stop, recover automatically, ignore the error or ask the user how to continue? \\
\hline ADOPTABILITY &
\summ{How does the system facilitate or obstruct adoption by both individuals and communities?}
Learnability &
TODO &
TODO \\
Sociability &
TODO &
TODO \\
\hline
\end{tabular}
\end{adjustwidth}
\caption{\label{summary-table} Quick reference sheet for our set of Technical Dimensions.}
\end{table}
