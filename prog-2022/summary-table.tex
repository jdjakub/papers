\newcommand{\summ}[1]{\multicolumn{2}{p{0.8\linewidth}}{#1} \\ \hline}
\setlength\extrarowheight{2pt}
\begin{table}
%\centering
\begin{tabular}{ >{\raggedleft\arraybackslash}p{0.2\linewidth}  p {0.4\linewidth}  >{\raggedright\arraybackslash}p{0.4\linewidth} }
\hline
Dimension (CLUSTER) & Summary & Range of key examples \\ \hline
\hline INTERACTION &
\summ{How do users execute their ideas, evaluate the result, and generate new ideas in response?}
Feedback Loops &
How wide are the various gulfs of \emph{execution} and \emph{evaluation} and how are they related? &
Immediate Feedback (short) vs. batch mode (long) gulf of evaluation \\
Modes of Interaction &
Which sets of feedback loops only occur together? &
Setup vs. editing vs. debugging \\
Abstraction Construction &
How do we go from abstractions to concrete examples and vice versa? &
Programming by Example vs. first principles \\
\hline NOTATION &
\summ{How are the different textual / visual programming notations related?}
Notational Structure &
What notations are used to program the system and how are they related? &
Notations overlap and need sync vs. complement each other \\
Surface / Internal Notations &
What is the connection between what a user sees and what a computer program sees? &
Sequence Editing vs. Rendering, Structure Editing vs. Recovery \\
Primary / Secondary Notations &
Is one notation more important than others? &
Secondary build scripts vs. visual editor and code on equal footing in Flash \\
Expression Geography &
Do similar expressions encode similar programs? &
Concise yet error-prone vs. explicit yet verbose \\
Uniformity of Notations &
Does the notation use a small or a large number of basic concepts? &
Lisp S-expressions vs. English-like textual notations \\
\hline ARCHITECTURE &
\summ{What is the system's attitude to its own organization and that of the outside world?}
Conceptual Outlook &
Does the system present as \emph{designed} or \emph{improvised}? &
Integrity (Everything is a X) vs. openness (compatible mixtures) \\
Composability &
What are the primitives? How can they be combined to achieve novel behaviors? &
Sequence, selection, repetition, function abstraction, recursion, logical connectives \\
Convenience &
Which wheels do users not need to reinvent? &
Small vs. expansive standard libraries \\
Commonality &
How much is \emph{common structure} explicitly marked as such? &
Common structure is redundantly flattened vs. factored out\\
\hline
\end{tabular}
\caption{\label{summary-table1} Quick reference sheet for our set of Technical Dimensions.}
\end{table}

\begin{table}
\begin{tabular}{ >{\raggedleft\arraybackslash}p{0.2\linewidth}  p {0.4\linewidth}  >{\raggedright\arraybackslash}p{0.4\linewidth} }
\hline
Dimension (CLUSTER) & Summary & Range of key examples \\ \hline
\hline CUSTOMIZABILITY &
\summ{Once a program exists in the system, how can it be extended and modified?}
Staging of Customization &
Must we customize \emph{running} programs differently to \emph{inert} ones? Do these changes last beyond termination? &
Source code vs. config files, Developer Tools tab, auto image-based persistence, scripting language \\
Externalizability &
Which portions of the system's state can be referenced and transferred to/from it? &
None (state is private) vs. all state exposed as human-legible, CSS-like addressing \\
Additive Authoring &
How far can the system's behavior be changed by \emph{adding} expressions? &
None (requires power to change original) vs. full (anything can be overridden repeatedly) \\
Self-Sustainability &
How far can the system’s behavior be changed from within? &
None (rely on extenal tools) vs. self-sufficient (contains everything needed) \\
\hline AUTOMATION &
\summ{How far does the system remove the need to spell out implementation in minute detail?}
Degrees of Automation &
What part of program logic does not need to be explicitly specified? &
Garbage collection (low-tech) vs. Prolog engine (hi-tech) \\
\hline ERRORS &
\summ{What does the system consider to be an error; how are they prevented and handled?}
Error Detection &
What errors can be detected in which feedback loops, and how? &
Human inspection in live coding vs. partial automation in static typing \\
Error Response &
How does the system respond when an error is detected? &
Does it stop, recover automatically, ignore the error or ask the user how to continue? \\
\hline ADOPTABILITY &
\summ{How does the system facilitate or obstruct adoption by both individuals and communities?}
Learnability &
What is the attitude towards the \emph{learning curve} and what is the target audience? &
HyperCard for the general public vs. FORTRAN for scientists \\
Sociability &
What are the social and economic factors that make the system the way it is? &
Funding, volunteers, code sharing, Q/A sites vs. documentation, sense of belonging \\
\hline
\end{tabular}
\caption{\label{summary-table2} Quick reference sheet for our set of Technical Dimensions.}
\end{table}
