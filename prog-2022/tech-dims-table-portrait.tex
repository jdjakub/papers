\newcommand{\summ}[1]{\multicolumn{2}{>{\hsize=\dimexpr2\hsize+2\tabcolsep+\arrayrulewidth\relax}X}{#1} \\ \hline}
%\newcommand{\summ}[1]{\\}

\setlength\extrarowheight{2pt}

\begin{table}
\caption{Technical Dimensions cheat sheet.}
\begin{tabularx}{\linewidth}{ >{\raggedleft\arraybackslash}p{3.75cm} X X }
\hline
Dimension (CLUSTER) & Summary & Range of key examples \\ \hline
\hline INTERACTION &
\summ{How do users manifest their ideas, evaluate the result, and generate new ideas in response?}
Feedback Loops &
How wide are the various gulfs of \emph{execution} and \emph{evaluation} and how are they related? &
Immediate Feedback (short) vs. batch mode (long) gulf of evaluation \\
Modes of Interaction &
Which sets of feedback loops only occur together? &
Setup vs. editing vs. debugging \\
Abstraction Construction &
How do we go from abstractions to concrete examples and vice versa? &
Programming by Example vs. first principles \\
\hline NOTATION &
\summ{How are the different textual / visual programming notations related?}
Notational Structure &
What notations are used to program the system and how are they related? &
Notations overlap and need sync vs. complement each other \\
Surface / Internal Notations &
What is the connection between what a user sees and what a computer program sees? &
Sequence Editing vs. Rendering, Structure Editing vs. Recovery \\
Primary / Secondary Notations &
Is one notation more important than others? &
Secondary build scripts vs. visual editor and code on equal footing in Flash \\
Expression Geography &
Do similar expressions encode similar programs? &
Concise yet error-prone vs. explicit yet verbose \\
Uniformity of Notations &
Does the notation use a small or a large number of basic concepts? &
Lisp S-expressions vs. English-like textual notations \\
\end{tabularx}
\label{tab:summary} 
\end{table}

\begin{table}
\caption{Technical Dimensions cheat sheet (continued).}
\begin{tabularx}{\linewidth}{ >{\raggedleft\arraybackslash}p{3.5cm} X X }
\hline
Dimension (CLUSTER) & Summary & Range of key examples \\ \hline
\hline CONCEPTUAL STRUCTURE &
\summ{How is meaning constructed? How are internal and external incentives balanced?}
Conceptual Integrity vs. Openness &
Does the system present as elegantly \emph{designed} or pragmatically \emph{improvised}? &
Integrity (Everything is a X) vs. openness (compatible mixtures) \\
Composability &
What are the primitives? How can they be combined to achieve novel behaviors? &
Sequence, selection, repetition, function abstraction, recursion, logical connectives \\
Convenience &
Which wheels do users not need to reinvent? &
Small vs. expansive standard libraries \\
Commonality &
How much is \emph{common structure} explicitly marked as such? &
Common structure is redundantly flattened vs. factored out \\
\hline CUSTOMIZABILITY &
\summ{Once a program exists in the system, how can it be extended and modified?}
Staging of Customization &
Must we customize \emph{running} programs differently to \emph{inert} ones? Do these changes last beyond termination? &
Source code vs. config files, Developer Tools tab, auto image-based persistence, scripting language \\
Externalizability / Additive Authoring &
Which portions of the system's state can be referenced and transferred to/from it? How far can the system's behavior be changed by \emph{adding} expressions? &
State is: (\emph{i}) private and hidden; (\emph{ii}) externalizable and overwriteable; (\emph{iii}) externalized and \emph{overridable} by addition \\
Self-Sustainability &
How far can the system’s behavior be changed from within? &
None (rely on external tools) vs. self-sufficient (contains everything needed) \\
\end{tabularx}
\label{tab:summary2}
\end{table}

\begin{table}
\caption{Technical Dimensions cheat sheet (continued).}
\begin{tabularx}{\linewidth}{ >{\raggedleft\arraybackslash}p{3.5cm} X X }
\hline
Dimension (CLUSTER) & Summary & Range of key examples \\ \hline
\hline COMPLEXITY &
\summ{How does the system structure complexity and what level of detail is required?}
Factoring of Complexity &
What programming details are hidden in reusable components and how? &
Domain-specific (more hiding) vs. general-purpose (less hiding) \\
Level of Automation &
What part of program logic does not need to be explicitly specified? &
Garbage collection (low-tech) vs. Prolog engine (hi-tech) \\
\hline ERRORS &
\summ{What does the system consider to be an \emph{error}? How are they prevented and handled?}
Error Detection &
What errors can be detected in which feedback loops, and how? &
Human inspection in live coding vs. partial automation in static typing \\
Error Response &
How does the system respond when an error is detected? &
Does it stop, recover automatically, ignore the error or ask the user how to continue? \\
\hline ADOPTABILITY &
\summ{How does the system facilitate or obstruct adoption by both individuals and communities?}
Learnability &
What is the attitude towards the \emph{learning curve} and what is the target audience? &
HyperCard for the general public vs. FORTRAN for scientists \\
Sociability &
What are the social and economic factors that make the system the way it is? &
Cathedral vs. Bazaar model \\
\end{tabularx}
\label{tab:summary3}
\end{table}
